%        File: TFGGuillermo.tex
%     Created: sáb mar 03 09:00  2018 C
% Last Change: sáb mar 03 09:00  2018 C
%
\documentclass[12pt,a4paper]{article}
\usepackage[utf8]{inputenc}
\usepackage[spanish, es-noquoting]{babel}
\usepackage[left=2.5cm,right=2.5cm,top=2.5cm,bottom=2.5cm]{geometry}
\usepackage{amsmath}
\usepackage{amsfonts}
\usepackage{amssymb}
\usepackage{amsthm, mathtools}
\usepackage{tikz,tikz-cd}
\usetikzlibrary{arrows, babel}
\usepackage{url}
\usepackage[colorlinks=true,linktocpage=true,pagebackref=true,linkcolor=blue]{hyperref}
\usepackage{remreset}
\usepackage{enumitem}
\usepackage{titlepic}
\usepackage{titlesec}
\usepackage{graphicx}
\usepackage{mathrsfs}
\usepackage{anyfontsize}

\renewcommand{\thesection}{\arabic{section}}

%Otro formato para las secciones
\titleformat{\section}[block]
{\fontsize{12}{15}\bfseries\sffamily\filcenter}
{\S \thesection.}
{1em}
{}

\makeatletter
\@removefromreset{section}{chapter}
\makeatother

%Márgenes
%\voffset=-1.5cm
%\hoffset=-1.5cm
%\setlength{\textwidth}{16cm}
%\setlength{\textheight}{23cm}
%\setlength{\headsep}{25pt}
%\footskip=30pt

%\parskip=1.2ex

%\usepackage{bm} Para poner los vectores de otra forma

%Fuente Palatino:
%\usepackage[sc]{mathpazo}
%Fuente Times:
\usepackage{newtxtext}
\usepackage{newtxmath}
%Fuente Libertine:
%\usepackage{libertine}
%\usepackage[libertine]{newtxmath}

%Para probar texto ciego:
%\usepackage{blindtext}

\newtheorem{thm}{Teorema}[section]
\newtheorem{prop}[thm]{Proposición}
\newtheorem{lema}{Lema}
\newtheorem{corol}[thm]{Corolario}
\theoremstyle{definition} \newtheorem{defn}[thm]{Definición}
\theoremstyle{definition} \newtheorem{ejemplo}[thm]{Ejemplo}
\theoremstyle{definition} \newtheorem{ejercicio}[thm]{Ejercicio}
\theoremstyle{remark} \newtheorem*{obs}{Observación}


\def\CC{\mathbb{C}}
\def\QQ{\mathbb{Q}}
\def\ZZ{\mathbb{Z}}
\def\RR{\mathbb{R}}
\def\TT{\mathbb{T}}
\def\SF{\mathbb{S}}
\def\NN{\mathbb{N}}
\def\dd{\mathrm{d}}
\def\lie{\mathscr{L}}
\def\gg{\mathfrak{g}}
\def\zz{\bar{z}}

%\newcommand{\vect}[1]{\bm{#1}}
\newcommand{\vect}[1]{\mathbf{#1}}
%\newcommand{\vect}[1]{\vec{#1}}

\newcommand{\parcial}[2]{\frac{\partial #1}{\partial #2}}
\newcommand{\pois}[2]{\left\lbrace#1,#2\right\rbrace}

\title{Simetrías en modelos integrables y superintegrables}
\author{Guillermo Gallego}
\date{\bf \today}

\begin{document}
%\title[SIMETRÍAS EN MODELOS INTEGRABLES Y SUPERINTEGRABLES]{
%{\rm\tiny Trabajo  \hspace{-1mm}de  \hspace{-1mm}Fin  \hspace{-1mm}de  \hspace{-1mm}Grado,  
%\hspace{-1mm}Junio  \hspace{-1mm}2018}\\[8pt] 
%SIMETRÍAS EN MODELOS INTEGRABLES Y SUPERINTEGRABLES\\[8pt]
%{\rm\tiny Departamento \hspace{-1mm}de  \hspace{-1mm}Física  \hspace{-1mm}Teórica\\
%Facultad  \hspace{-1mm}de  \hspace{-1mm}Ciencias \hspace{-1mm}Físicas,  \hspace{-1mm}UCM\\
%{\rm\tiny Dirigido por Miguel A. Rodríguez}
%}}
\maketitle
\begin{abstract}
  El abstract
\end{abstract}
\newpage

\tableofcontents

\section*{Introducción}
\addcontentsline{toc}{section}{Introducción}
Aquí la introducción.
\section{Conceptos básicos: geometría simpléctica, simetrías e integrabilidad}
La estructura matemática característica de los espacios de fases en Mecánica Clásica es la de \emph{variedad simpléctica}. Una variedad simpléctica es un par $(M,\omega)$, donde $M$ es una variedad diferenciable y $\omega$ es una $2$-forma diferencial (un tensor covariante antisimétrico de grado $2$), no degenerada y cerrada, es decir, tal que $\dd \omega=0$. En el caso en que la forma $\omega$ sea exacta, es decir, que exista una $1$-forma $\theta$ tal que $\omega=\dd \theta$, se dice que la variedad $(M,\omega)$ es exacta. 

El ejemplo más básico de variedad simpléctica es el espacio $\RR^{2n}=\left\{ (\vect{q},\vect{p}) \right\}$, con $\vect{q}=(q_1,\dots,q_n)$ y $\vect{p}=(p_1,\dots,p_n)$, equipado con la forma $\omega=\dd p_i \wedge \dd q_i$. Claramente esta forma es exacta pues $\omega=\dd \theta$, con $\theta=p_i \dd q_i$. El \emph{teorema de Darboux}, garantiza que para toda variedad simpléctica $(M,\omega)$ localmente es posible encontrar unas coordenadas $(\vect{q},\vect{p})$ en las que la forma toma el aspecto $\omega=\dd p_i \wedge \dd q_i$. Este tipo de coordenadas se llaman \emph{coordenadas de Darboux}.

La forma $\omega$ induce una dualidad entre campos y $1$-formas. Resulta que a cada campo vectorial $\xi$ en la variedad le podemos asignar la forma $i_{\xi}\omega$, donde $i$ denota la \emph{contracción}, es decir $i_{\xi}\omega=\omega(\xi,\bullet)$. Un campo $\xi$ se dice \emph{simpléctico} si $\lie_{\xi}\omega=0$, donde $\lie$ denota la \emph{derivada de Lie}. Ahora, por la fórmula de Cartan, $\lie_{\xi}\omega=i_{\xi}(\dd \omega)+\dd(i_{\xi}\omega)=\dd(i_{\xi}\omega)$, de modo que un campo $\xi$ es simpléctico si y sólo si la forma $i_{\xi}\omega$ es cerrada. En el caso particular en que esta forma $i_{\xi}\omega$ sea exacta, existirá una función $H:M\rightarrow \RR$ tal que $\dd H=-i_{\xi}\omega$ y decimos que $\xi$ es un \emph{campo hamiltoniano} de \emph{hamiltoniano} $H$ y se denota $\xi_H$. Un cálculo sencillo muestra que, localmente, en coordenadas de Darboux, las curvas integrales del campo $\xi_H$ seguirán las \emph{ecuaciones de Hamilton}
\begin{equation}
  \begin{cases}
    \dot{q_i}=\parcial{H}{p_i}\\ 
    \dot{p_i}=-\parcial{H}{q_i} ,
  \end{cases}
  \label{eq:hamilton}
\end{equation}
con $i=1,\dots,n$.
Así, entenderemos por un \emph{sistema hamiltoniano} con $n$ grados de libertad una función $H:M\rightarrow \RR$ definida sobre una variedad simpléctica $M$ de dimensión $2n$.

El \emph{corchete de Poisson} también puede recuperarse ahora en el contexto simpléctico definiendo simplemente $\left\{ F,G \right\}=\xi^F G$, que localmente, en coordenadas de Darboux, se expresará en la forma clásica
\begin{equation}
  \left\{ F,G \right\}=\parcial{F}{p_i}\parcial{G}{q_i}-\parcial{F}{q_i}\parcial{G}{p_i}.
  \label{eq:poisson}
\end{equation}
En este contexto las transformaciones canónicas serán simplemente difeomorfismos $\varphi:M\rightarrow M$ que preserven la forma $\omega$, es decir, tales que $\varphi^*\omega=\omega$ o, equivalentemente que preserven el corchete de Poisson, $\left\{ F,G \right\}\circ \varphi=\left\{ F\circ \varphi,G\circ \varphi \right\}$.

En el formalismo simpléctico el teorema de Noether adquiere también un nuevo cariz. Para ver esto supongamos que la variedad $M$ es conexa y consideremos un grupo de Lie $G$ que actúa sobre $M$ de tal forma que para cada $g\in G$, la aplicación $\Phi_g:M\rightarrow M$ inducida por la acción es una transformación canónica. En tal caso decimos que la acción $\Phi:G\times M\rightarrow M$ es una \emph{acción simpléctica}. Consideramos ahora una función $J:M\rightarrow \gg^*$, donde $\gg^*$ denota el dual del álgebra de Lie $\gg$ de $G$ y, para cada $\xi\in \gg$, llamamos $\hat{J}$ a la función
\begin{align*}
  \hat{J}(\xi) :M&\longrightarrow \RR\\ 
     x &\longmapsto J(x)\cdot \xi.
  \end{align*}
  Decimos entonces que $J$ es una \emph{aplicación momento} para la acción si $\dd \hat{J}(\xi)=-i_{\xi_M}\omega$, donde $\xi_M$ denota el generador infinitesimal de la acción correspondiente a $\xi$ (es decir, de la inducida por el subgrupo uniparamétrico $\exp(\xi t)$). Nótese también que el campo de hamiltoniano $\hat{J}(\xi)$ es precisamente $\xi_M$. Formulamos entonces la «versión simpléctica» del teorema de Noether 
  \begin{thm}[Noether]
    Sea $\Phi:G\times M \rightarrow M$ una acción simpléctica de un grupo de Lie $G$ en una variedad simpléctica $(M,\omega)$ con una aplicación momento $J$. Supongamos que $H:M\rightarrow \RR$ es invariante bajo la acción, esto es, $H(x)=H(\Phi_g(x))$ para cualesquiera $x\in M$ y $g\in G$. Entonces $J$ es una cantidad conservada del campo hamiltoniano $\xi_H$, esto es, si $\varphi_t$ es el flujo de $\xi_H$, $J(\varphi_t(x))=J(x)$ para todo $t$.
  \end{thm}

  La mejor forma de entender esto es ilustrarlo con un ejemplo:
  \begin{ejemplo}[Conservación del momento angular]
    Consideremos el grupo de rotaciones $\mathrm{SO}(3)$ actuando de forma simpléctica sobre el espacio de fases $\RR^6=\RR^3\times \RR^3$ en la forma
    \begin{align*}
      \mathrm{SO}(3)\times \RR^6&\longrightarrow \RR^6\\ 
      (R,(\vect{q},\vect{p})) &\longmapsto (R\vect{q},R\vect{p}) .
      \end{align*} 
      Los elementos de su álgebra de Lie $\mathfrak{so}(3)$ son los generadores infinitesimales de las rotaciones que, como es bien sabido, pueden asociarse con operadores de la forma $J_{\vect{u}}=\vect{u}\times \bullet$, con $u\in \RR^3$ un vector cuya dirección es la del eje de la rotación y su norma la velocidad angular del giro. Así, a cada $\vect{u}\in \mathfrak{so}(3)$ podemos asociarle el campo en $M$ de la forma $\vect{u}_M=(\vect{u}\times \vect{q}, \vect{u}\times \vect{p})$, que en coordenadas se escribe
      \begin{equation*}
	\vect{u}_M=\epsilon_{ijk}u_k(q_i \partial_{q_j}+p_i \partial_{p_j}).
      \end{equation*}
      Ahora, si consideramos el momento angular $\vect{L}(\vect{q},\vect{p})=\vect{q}\times \vect{p}$, que en coordenadas se expresa $L_k(q_i,p_j)=\epsilon_{ijk}q_ip_j$, de modo que $(\vect{L}\times\vect{u})_{k}=\epsilon_{ijk}u_kq_ip_j$ y 
      \begin{align*}
	\dd(\vect{L}\times \vect{u})&=\epsilon_{ijk}u_k(q_i\dd p_j + p_j\dd q_i)=\epsilon_{ijk}u_k(q_i\dd p_j - p_i\dd q_j)=-i_{\vect{u}_M}\omega.
      \end{align*}
      Por tanto, si consideramos la aplicación $L:\RR^6\rightarrow\mathfrak{so}(3)^*$ que a cada $(\vect{q},\vect{p})$ le asigna  
      \begin{align*}
	L(\vect{q},\vect{p}) :\mathfrak{so}(3)&\longrightarrow \RR\\ 
	\vect{u} &\longmapsto \vect{L}(\vect{q},\vect{p})\times \vect{u},
	\end{align*}
	tenemos que $L$ es una aplicación momento de la acción de $\mathrm{SO}(3)$. Por el teorema de Noether, si consideramos un hamiltoniano $H$ en $\RR^6$ que sea invariante bajo rotaciones, tenemos que la aplicación momento $L$ es una integral primera del sistema hamiltoniano dado por $H$. Como consecuencia, el momento angular $\vect{L}$ es una cantidad conservada del sistema.
  \end{ejemplo}

  Finalmente, uno de los aspectos de la Mecánica Clásica donde el formalismo simpléctico muestra todo su potencial es la teoría de sistemas integrables. 
  \begin{defn}
    Un sistema hamiltoniano $H$ sobre una variedad simpléctica $(M,\omega)$ de dimensión $2n$ se dice \emph{completamente integrable} si admite $n$ integrales primeras $F_1=H,\dots,F_n$ en involución, es decir, tales que
    \begin{equation}
      \left\{ F_i,F_j \right\}=0,  
    \end{equation}
    para cualesquiera $i,j=1,\dots,n$ y funcionalmente independientes, es decir,
    \begin{equation}
      \dd F_{1,x} \wedge \cdots \wedge \dd F_{n,x}\neq 0
    \end{equation}
    para casi todo punto $x\in M$.
  \end{defn}

  El teorema central de toda la teoría de sistemas integrables es el siguiente:
  \begin{thm}[Arnold-Liouville]\label{arnoldliouville}
   Sea un sistema hamiltoniano $H$ sobre una variedad simpléctica $(M,\omega)$ de dimensión $2n$ que es completamente integrable y sea $F=(F_1,\dots,F_n):M\rightarrow \RR^n$ con $F_1,\dots,F_n$ las integrales primeras en involución del sistema. Entonces:
   \begin{enumerate}
     \item Los conjuntos de nivel $F^{-1}(a)$ son subvariedades de $M$ invariantes bajo el flujo del sistema,
     \item Las componentes conexas de los conjuntos de nivel son difeomorfas a $\TT^k\times \RR^{n-k}$ para cierto $0\leq k \leq n$, donde $\TT^k$ denota el toro $k$-dimensional $\TT^k=\SF^1\times \cdots \times \SF^1$, $k$ veces. En particular, las compactas son difeomorfas a toros $\TT^n$, que llamamos \emph{toros de Liouville}. Además, en cada uno de estos conjuntos de nivel la dinámica es lineal, es decir, existen unas coordenadas $\vect{w}$ en estos conjuntos tales que $\dot{\vect{w}}=\mathbf{cte}$.
     \item En torno a cada toro de Liouville podemos tomar un entorno difeomorfo a un producto de toros de Liouville en los que podemos dar unas coordenadas de Darboux $(\vect{J},\vect{w})$, llamadas \emph{variables de acción-ángulo}, tales que las $\vect{J}$ son constantes en cada toro de Liouville y las $\vect{w}$ son coordenadas angulares en esos toros. Como consecuencia, en estos entornos las ecuaciones de Hamilton son integrables por cuadraturas.
   \end{enumerate}
  \end{thm}
  En resumen este teorema nos dice que si un sistema hamiltoniano tiene una cantidad suficiente de integrales primeras (o de simetrías, aunque éstas no son siempre evidentes) el comportamiento de este sistema será muy sencillo. Esto es una motivación suficiente para buscar sistemas integrables, tal vez mediante la metodología de hallar sus grupos de simetría y asociarles sus aplicaciones momento. Concluimos la sección dando un ejemplo clásico de sistema integrable.
  \begin{ejemplo}[Potencial central]
    Consideramos el caso genérico de una partícula moviéndose en el espacio tridimensional sujeta a un potencial central, $V(\vect{q},\vect{p})=V(r)$, con $r=|\vect{q}|$. El espacio de fases será $\RR^6=\left\{ (\vect{q},\vect{p}) \right\}$ y el hamiltoniano del sistema vendrá dado por
    \begin{equation}
      H(\vect{q},\vect{p})=\frac{\vect{p}^2}{2m}+V(r).
      \label{eq:central}
    \end{equation}
    nomo las rotaciones preservan el producto escalar, el sistema será invariante bajo rotaciones y, por tanto, el momento angular $\vect{L}(\vect{q},\vect{p})=\vect{q}\times \vect{p}$ es una cantidad conservada del sistema. En particular serán cantidades conservadas $L^2=\vect{L}\cdot \vect{L}$ y $L_3=q_1p_2-q_2p_1$ la componente vertical de $\vect{L}$. Ahora, si calculamos el corchete de Poisson
  \begin{equation*}
    \pois{L^2}{L_3}=\pois{L_i^2}{L_3}=2L_i\pois{L_i}{L_3},
  \end{equation*}
  por la regla de Leibniz. Recordando las reglas de conmutación del momento angular, $\left\{ L_i,L_j \right\}=\epsilon_{ijk}L_k$, tenemos
\begin{equation*}
  \pois{L^2}{L_3}=-2L_1L_2+2L_2L_1=0.
\end{equation*}
Por tanto, $H$, $L^2$ y $L_3$ son $3$ funciones en involución, de modo que el potencial central es completamente integrable.

  \end{ejemplo}
  \section{Sistemas superintegrables, trayectorias cerradas y simetrías ocultas}
  Podemos considerar ahora qué sucedería si un sistema integrable con $n$ grados de libertad tuviera cantidades conservadas adicionales, hasta otras $n-1$ para que puedan existir trayectorias, funcionalmente independientes de las anteriores, aunque ya no podrían estar en involución con todas. Esto motiva entonces la noción de \emph{superintegrabilidad} o \emph{integrabilidad no abeliana}.
  \begin{defn}
    Un sistema hamiltoniano $H$ sobre una variedad simpléctica $(M,\omega)$ de dimensión $2n$ se dice \emph{superintegrable} si admite $n+k$ integrales primeras funcionalmente independientes para cierto $k=1,\dots,n-1$. En el caso en que $k=n-1$ el sistema se dice \emph{maximalmente superintegrable}.
  \end{defn}

  Las dos primeras partes del teorema \ref{arnoldliouville} se pueden generalizar al caso superintegrable, es lo que se conoce como el \emph{teorema de Mishchenko-Fomenko}. Destaca especialmente el caso maximalmente superintegrable, para el cual el resultado nos dice que las órbitas serán curvas cerradas con movimiento periódico.

  Vamos a aplicar estas ideas al caso del potencial central, para ver en qué casos el sistema será superintegrable. Como ya hemos visto, en el movimiento en un potencial central, el momento angular $\vect{L}=\vect{q}\times \vect{p}$ se conserva. Puesto que $\vect{L}$ es perpendicular a $\vect{q}$ y a $\vect{p}$, la conservación de $\vect{L}$ significa que, durante todo el movimiento, la posición y el momento de la partícula permanecen en un mismo plano, perpendicular a $\vect{L}$. La invariancia bajo rotaciones nos permite escoger la dirección $\vect{e}_z$ paralela al momento angular $\vect{L}$ y estudiar el sistema «reducido» en el plano perpendicular, que ahora será el plano $XY$. Llamando ahora $r=x^2+y^2$, el hamiltoniano en este «sistema reducido» será
  \begin{equation}
    H(x,y,p_x,p_y)= \tfrac{1}{2m}(p_x^2+p_y^2)+V(r).
  \end{equation}
  Cambiando a coordenadas polares $\vect{q}=(x,y)=r\vect{e}_{r}$, con $\vect{e}_r=(\cos \phi, \sin \phi)$, $p_r=m\dot r$, $p_{\phi}=mr^2\dot \phi$, tenemos
  \begin{equation}
    H(r,\phi,p_r,p_{\phi})=\frac{p_r^2}{2m}+\frac{p_{\phi}^2}{2mr^2}+V(r). 
  \end{equation}
  Nótese ahora que, si $\vect{p}=m\dot{\vect{q}}=m\dot r \vect{e}_r + mr \dot \phi \vect{e}_{\phi}$, con $\vect{e}_{\phi}=(-\sin \phi,\cos \phi)$, entonces $\vect{L}=\vect{q}\times \vect{p}=mr^2\dot \phi \vect{e}_z$, luego $L=|\vect{L}|=p_{\phi}$. Por tanto, el hamiltoniano queda en la forma
  \begin{equation}
    H(r,\phi,p_r,p_{\phi})=\frac{p_r^2}{2m}+\frac{L^2}{2mr^2}+V(r). 
  \end{equation}
  Al haber restringido el sistema a un plano y quedarnos con sólo dos grados de libertad, la integrabilidad sigue garantizada por la conservación del momento angular, mientras que la existencia de una cantidad conservada adicional independiente del momento angular nos daría la superintegrabilidad.

  Las trayectorias de energía $E$ y momento angular $L$ verificarán entonces las ecuaciones
  \begin{equation}
    \begin{cases}
    E=\tfrac{1}{2}m\dot r^2+\frac{L^2}{2mr^2}+V(r) \\
    L=mr^2\dot \phi.
  \end{cases}
  \end{equation}
  De modo que
  \begin{equation}
      \dot r=\sqrt{\frac{2}{m}[E-V(r)]-\frac{L^2}{2mr^2}} 
  \end{equation}
  y 
  \begin{equation}
    t=\int \frac{dr}{\sqrt{\frac{2}{m}[E-V(r)]-\frac{L^2}{2mr^2}}}.
  \end{equation}
  Por tanto, el ángulo $\phi$ puede ser integrado por cuadraturas en la forma
  \begin{equation}
    \phi=\int\frac{Ldr}{r^2\sqrt{2m[E-V(r)]-L^2/r^2}}. 
    \label{eq:intphi}
  \end{equation}

Observemos que podemos considerar la partícula sujeta a un «potencial efectivo»
\begin{equation}
  U=\frac{L^2}{2mr^2}+V(r),
\end{equation}
donde el término $\frac{L^2}{2mr^2}$ constituye una «energía centrífuga». En los valores de $r$ para los cuales $U$ es constante la velocidad radial $\dot r$ se anula y nos indica los \emph{puntos de retorno} de la trayectoria. En el caso en que el movimiento sea acotado, $r$ varía entre dos límites $r_{\text{min}}$ y $r_{\text{máx}}$ y la trayectoria está contenida enteramente en el interior de una corona circular limitada por las circunferencias de radios $r_{\text{min}}$ y $r_{\text{máx}}$. Sin embargo, esto no implica que la trayectoria sea una curva cerrada. En efecto, si, de acuerdo con \eqref{eq:intphi}, calculamos el ángulo que gira el vector posición en lo que la partícula se mueve entre $r_{\text{min}}$ y $r_{\text{máx}}$ obtenemos
\begin{equation}
  \Delta \phi = 2\int_{r_{\text{min}}}^{r_{\text{máx}}} \frac{Ldr}{r^2\sqrt{2m[E-V(r)]-L^2/r^2}}.
\end{equation}
La condición que ha de cumplirse para que la trayectoria en cierto momento se cierre es precisamente que este ángulo sea un múltiplo racional de $2\pi$, es decir, que $\Delta \phi = \frac{2\pi m}{n}$, para ciertos enteros $m$ y $n$. En un caso general esta condición no se cumple y la trayectoria va llenando densamente toda la corona en la que se encuentra contenida [figura].

J. Bertrand probó en 1883 que precisamente los únicos potenciales centrales en los cuales todos los movimientos acotados se dan en trayectorias cerradas son aquellos en los que $V(r)$ es proporcional a $1/r$ (potencial de Kepler) o a $r^2$ (oscilador armónico isótropo). Por tanto, en el resto de casos el sistema no es superintegrable. Cabe preguntarse entonces si tanto el potencial de Kepler como el oscilador armónico isótropo tendrán alguna integral primera adicional que nos garantice la superintegrabilidad, tal vez dada por una «simetría oculta» del sistema, es decir, por una simetría que no era aparente a primera vista.

La respuesta es afirmativa en los dos casos. En el potencial de Kepler $V(r)=k/r$ existe una cantidad conservada adicional que es el \emph{vector de Laplace-Runge-Lenz}, definido por
\begin{equation}
  \vect{A}=\vect{p}\times \vect{L}-mk\frac{\vect{q}}{r}. 
\end{equation}
En el caso del oscilador armónico isótropo $V(r)=kr^2$ existe un tensor simétrico conservado, el \emph{tensor de Fradkin}, dado por
\begin{equation}
  A_{ij}=\tfrac{1}{2m}(p_ip_j+kq_iq_j), 
\end{equation}
con $i,j=1,2$. La traza del tensor es precisamente la energía $\tfrac{1}{2m}(\vect{p}^2+kr^2)$ y la componente de fuera de la diagonal $F=A_{12}=\tfrac{1}{2m}(p_1p_2+kq_1q_2)$ es precisamente la cantidad conservada adicional, indepentiende de la energía y del momento angular. 

Si se regresa al espacio tridimensional y se consideran ahí las cantidades conservadas adicionales ahora obtenidas, se pueden calcular las relaciones entre las diferentes integrales primeras del sistema con el corchete de Poisson y ver qué subalgebra de Lie del álgebra total de las variables dinámicas del sistema generan, desvelando así el verdadero rostro de estas «simetrías ocultas». Se puede probar que para el caso del potencial de Kepler esta álgebra es $\mathfrak{so}(4)$ mientras que para el caso del oscilador armónico isótropo es $\mathfrak{su}(3)$.

Para terminar la sección, cabe comentar que la superintegrabilidad de un sistema clásico también tiene sus consecuencias en Mecánica Cuántica. Por ejemplo, en el caso del átomo de hidrógeno (que se puede considerar el análogo cuántico del problema de Kepler, de modo que conserva las mismas simetrías) la existencia de la cantidad conservada adicional, el vector de Laplace-Runge-Lenz, es la causante de la degeneración «accidental» de sus niveles de energía. Por degeneración accidental de los niveles del átomo de hidrógeno nos referimos al hecho de que sus niveles de energía dependan tan solo del primer número cuántico, $n$, y no del número cuántico $l$.

\section{El oscilador armónico anisótropo}
En esta sección estudiamos la superintegrabilidad de otro sistema clásico, el oscilador armónico anisótropo. Para encontrar las integrales primeras adicionales nos ayudaremos de unas coordenadas complejas adicionales, que nos permitirán encontrar un tensor conservado. Este procedimiento se debe a Jauch y Hill [citar] y sus ideas principales nos serán útiles más adelante para tratar otros problemas.

El oscilador armónico anisótropo con $n$ grados de libertad está definido sobre el espacio de fases $\RR^{2n}=\left\{ \vect{q},\vect{p} \right\}$ y su hamiltoniano viene dado por
\begin{equation}
  H(\vect{q},\vect{p})=\frac{1}{2}\sum_{i=1}^n \left(p_i^2 + \omega_i^2q_i^2\right)=\sum_{i=1}^n H_i(q_i,p_i),
  \label{eq:anisotropo}
\end{equation}
con $H_i(q_i,p_i)=\frac{1}{2}(p_i^2+\omega_i^2q_i^2)$. Para probar la integrabilidad del sistema basta tomar las funciones $F_j=\sum_{i=j}^n H_i$, con $j$ recorriendo desde $1$ hasta $n$, ya que
\begin{equation*}
  \left\{ H_i,H_j \right\}=\parcial{H_i}{p_k}\parcial{H_j}{q_k}-\parcial{H_j}{p_k}\parcial{H_i}{q_k}=0.
\end{equation*}

Los conjuntos de nivel de las funciones $F_j(\vect{q},\vect{p})=a_j$ serán toros de Liouville dados por las ecuaciones
\begin{equation*}
  \begin{cases}
    p_1^2+\omega_1^2q_1^2&= 2(a_1-a_2)\\
    p_1^2+\omega_2^2q_1^2&= 2(a_2-a_3)\\
     &\vdots \\
    p_n^2+\omega_n^2q_n^2&= 2a_n.
  \end{cases}
\end{equation*}
Es posible demostrar que las trayectorias del flujo hamiltoniano en estos toros son cerradas si y sólo si las frecuencias $\omega_i$ son de la forma $\omega_i=n_i \omega$, para cierta $\omega$ y para ciertos números enteros $n_i$, de modo que el hamiltoniano \eqref{eq:anisotropo} toma la forma
\begin{equation}
  H(\vect{q},\vect{p})=\frac{1}{2}\sum_{i=1}^n \left(p_i^2 + n_i^2\omega^2q_i^2\right).
\end{equation}
Veamos entonces que precisamente en estos casos el oscilador armónico anisótropo es superintegrable.

Para encontrar las integrales primeras adicionales vamos a servirnos de un espacio de fases complejo auxiliar $\CC^n$, con coordenadas $z_i, \bar{z}_i$. Concretamente, definimos
\begin{equation}
  z_j=p_j-in_j\omega q_j,\ \ \  \bar{z}_j=p_j+in_j\omega q_j.  
\end{equation}
Tenemos entonces que $H_i=\frac{1}{2}z_i\bar{z}_i$ y el hamiltoniano es de la forma
\begin{equation}
  H=\frac{1}{2}\sum_{i=1}^n z_i\bar{z}_i. 
\end{equation}
Ahora, se comprueba fácilmente que las cantidades
\begin{equation}
  c_{ij}=z_i^{n_j}\bar{z}_j^{n_i} 
\end{equation}
son integrales primeras del sistema. En el caso en que $n_i=n_j$, entre estas cantidades tenemos los momentos angulares
\begin{equation}
  L_{ij}=q_ip_j-q_jp_i 
\end{equation}
y un tensor similar al de Fradkin
\begin{equation}
  A_{ij}=p_ip_j+n_in_j\omega^2 q_iq_j. 
\end{equation}

\section{El hamiltoniano de TTW}
Principalmente por consideraciones al cuantizar, dentro de los sistemas integrables y superintegrables son de especial interés aquellos en los que sus integrales primeras son polinomiales en los momentos. Así, estos sistemas se llamarán \emph{polinomialmente integrables} (respectivamente, \emph{polinomialmente superintegrables}) y llamaremos \emph{orden} de sus integrales primeras al grado que estas tengan como polinomios en los momentos.

Los sistemas superintegrables de segundo orden han sido bastante estudiados y poseen una teoría de clasificación. También han sido tratados varias clases de sistemas de tercer orden. Sin embargo, para sistemas generales de tercer orden y de órdenes superiores se conoce mucho menos. En particular hasta hace unos pocos años había pocos ejemplos y las teorías de estructura y clasificación eran prácticamente inexistentes. 

La situación cambió drásticamente en 2009 con el trabajo de F. Tremblay, A. V. Turbiner y P. Winternitz [citar], donde los autores presentaron una familia de potenciales en el plano, parametrizados por una constante $k$, y conjeturaron que estos sistemas eran superintegrables para todo $k\in \QQ$, con órdenes tan grandes como se quisieran. Un año más tarde, en otro trabajo [citar] calcularon las trayectorias acotadas de esta familia de sistemas y mostraron que eran curvas cerradas para todos los casos con $k \in \QQ$, dando así evidencia a favor de su conjetura. Posteriormente se verificó que, efectivamente, sus conjeturas eran correctas.

Concretamente, el hamiltoniano de Tremblay, Turbiner y Winternitz (que a partir de ahora abreviaremos como \emph{hamiltoniano de TTW}) en coordenadas polares tiene la forma 
\begin{equation*}
  H_k(r,\phi,p_r,p_{\phi})=\frac{1}{2}\left( p_r^2+\frac{p^2_{\phi}}{r^2} +\omega^2 r^2 \right) + V_k(r,\phi),
\end{equation*}
con
\begin{equation}
  V_k(r,\phi)=\frac{\alpha k^2}{2r^2\cos^2 k\phi}+\frac{\beta k^2}{2r^2\sin^2 k\phi}.
  \label{eq:TTW}
\end{equation}
La integrabilidad viene garantizada por la integral primera
\begin{equation}
  X_k=L^2+\frac{\alpha k^2}{r^2\cos^2 k\phi}+\frac{\beta k^2}{r^2\sin^2 k\phi},
\end{equation}
donde $L=p_{\phi}$ es el momento angular. La existencia de una integral primera adicional $Y_{2k}$ de orden $2k$ para $k\in \QQ$ que diera la superintegrabilidad del sistema fue conjeturada por Tremblay, Turbiner y Winternitz en [citar] y posteriormente demostrada.

\section{El hamiltoniano de TTW en coordenadas cartesianas}
Para continuar con el estudio del Hamiltoniano de TTW, en esta sección vamos a ver cómo se expresa en coordenadas cartesianas y vamos a representar el potencial en los casos más sencillos.

En coordenadas cartesianas, el hamiltoniano de TTW toma la forma
\begin{equation*}
  H_k(x,y,p_x,p_y)=\frac{1}{2}[p_x^2+p_y^2+\omega^2(x^2+y^2)]+V_k(x,y),
\end{equation*}
a falta de hallar el aspecto del potencial $V_k$ en coordenadas cartesianas.
Teniendo en cuenta que $(x,y)=r\vect{e}_r$, en coordenadas cartesianas el potencial queda
\begin{align*}
  V_k(x,y)=\frac{\alpha k^2}{2r^2\cos^2(k\arctan \tfrac{y}{x})} +\frac{\beta k^2}{2r^2\sin^2(k\arctan \tfrac{y}{x})}
\end{align*}
o, equivalentemente,
\begin{align*}
  V_k(x,y)=\frac{k^2}{2(x^2+y^2)}\left( \alpha\left[ 1+\tan^2\left( k\arctan\frac{y}{x} \right) \right] \right. \left. +\beta\left[ 1+\cot^2\left( k\arctan\frac{y}{x} \right) \right]\right).
\end{align*}
Esta expresión solo es sencilla para algunos valores de $k$, por ejemplo, para $k=1$,
\begin{equation*}
  V_1=\frac{1}{2}\left( \frac{\alpha}{x^2}+\frac{\beta}{y^2} \right).
\end{equation*}
Para valores enteros pequeños de $k$ podemos hacer algunas simplificaciones y obtenemos
\begin{equation*}
  V_2=\frac{1}{2}(x^2+y^2)\left( \frac{4\alpha}{(x^2-y^2)^2}+\frac{\beta}{x^2y^2} \right)
\end{equation*}
y
\begin{equation*}
  V_3=\frac{9}{2}(x^2+y^2)^2\left( \frac{\alpha}{(x^3-3xy^2)^2}+\frac{\beta}{(-3x^2y+y^3)^2} \right).
\end{equation*}
Destaca también el caso con $k=\tfrac{1}{2}$,
\begin{equation*}
  V_{1/2}=\frac{1}{2y^2}\left[ \alpha\left( 1-\frac{x}{\sqrt{x^2+y^2}} \right)+ \beta\left( 1+\frac{x}{\sqrt{x^2+y^2}} \right) \right].
\end{equation*}

\section{El hamiltoniano de TTW en coordenadas complejas}
Inspirados por el cambio a coordenadas complejas que hemos hecho al tratar el oscilador armónico anisótropo, vamos a tratar de introducir unas coordenadas complejas al caso del hamiltoniano TTW, para ver si lo podemos dejar en una forma que nos dé más información sobre su comportamiento. En este caso, sin embargo, el cambio propuesto es ligeramente distinto, ya que no queremos combinar las coordenadas con los momentos sino que queremos obtener unas nuevas coordenadas generalizadas complejas y ver qué aspecto tendrán sus momentos canónicos conjugados. Concretamente, queremos hacer un cambio de esta forma
\begin{equation}
  z=x+iy.
\end{equation}
Esta clase de tratamientos de las coordenadas generalizadas son realizados comúnmente en física en diversas situaciones, sin embargo, cabe detenerse por un momento para entender qué es lo que está sucediendo geométricamente.

Sea $M$ una variedad diferenciable. Una \emph{estructura cuasicompleja} en $M$ es un endomorfismo
\begin{align*}
  J :TM&\longrightarrow TM
  \end{align*}
  tal que $J^2=-\mathrm{id}$. Nótese que por tanto $M$ ha de tener dimensión par. Podemos considerar entonces el \emph{fibrado tangente complexificado}
  \begin{equation*}
    T_{\CC}M=TM\otimes \CC.
  \end{equation*}
Ahora, si consideramos la aplicación bilineal
\begin{align*}
  \tilde{J} :TM \times \CC&\longrightarrow TM\otimes \CC\\ 
    (\xi,z) &\longmapsto J(\xi)\otimes \CC, 
  \end{align*}
  entonces, por la propiedad universal del producto tensorial existe una única aplicación lineal $I:T_{\CC}M \rightarrow T_{\CC}M$ tal que el siguiente diagrama conmuta
  \begin{center}
    \begin{tikzcd}
      TM \times \CC      \arrow{rr}{\otimes}\arrow{rrdd}[anchor=north,rotate=-30]{\tilde{J}} && T_{\CC}M = TM \otimes \CC \arrow{dd}[anchor=west]{I} \\ 
       && \\
       &&T_{\CC}M.
     \end{tikzcd}
   \end{center}
   La aplicación $I$ es entonces un endomorfismo de $T_{\CC}M$ con autovalores $\pm i$ y podemos descomponer
   \begin{equation*}
     T_{\CC}M=\ker(I-i)\oplus \ker(I+i).
   \end{equation*}

   En el caso particular en que $M$ se una variedad de dimensión $2$ con coordenadas $(x,y)$, podemos considerar entonces la «coordenada compleja» 
   \begin{equation*}
     z=x+iy.
   \end{equation*}
   Ahora, en el fibrado tangente $TM$ podemos considerar la base holónoma $\left\{ \partial_x, \partial_y \right\}$ y la estructura cuasicompleja dada por
   \begin{align*}
     I :TM&\longrightarrow TM\\ 
       \partial_x &\longmapsto \partial_y \\
       \partial_y &\longmapsto -\partial_x .
     \end{align*}
     Consideremos entonces el fibrado tangente complexificado $T_{\CC}M$ y los vectores
     \begin{equation*}
       \begin{cases}
	 \partial_z=\tfrac{1}{2}(\partial_x-i\partial_y) \\
	 \partial_{\bar{z}}=\tfrac{1}{2}(\partial_x+i\partial_y) .
       \end{cases}
     \end{equation*}
     Es inmediato comprobar que 
	 $\partial_z\in \ker(I-i)$ y que
	 $\partial_{\bar{z}} \in \ker(I+i)$,
	 de modo que $\left\{ \partial_z, \partial_{\bar{z}} \right\}$ es una base de $T_{\CC}M$. Si tenemos un vector $v \in TM$ que en la base holónoma original se escribe
     \begin{equation*}
       v = \dot{x} \partial_x + \dot{y} \partial_y,
     \end{equation*}
     entonces, si pensamos en $v$ como vector de $T_{\CC}M$, podemos escribirlo en la base $\left\{ \partial_z, \partial_{\bar{z}} \right\}$ en la forma
     \begin{equation*}
       v= \dot{z} \partial_z + \dot{\bar{z}} \partial_{\bar{z}},
     \end{equation*}
     con $\dot{z}=\dot{x} + i \dot{y}$. De modo que un lagrangiano $\mathcal{L}:TM\rightarrow \RR$, que localmente puede expresarse en las coordenadas originales
     \begin{equation*}
       \mathcal{L}(v)=\mathcal{L}(x,y,\dot{x},\dot{y}),
     \end{equation*}
     puede extenderse a $T_{\CC}M$ en las nuevas coordenadas complejas en la forma
     \begin{equation}
       \mathcal{L} (v)=\mathcal{L}(z,\bar{z},\dot{z},\dot{\bar{z}}).
     \end{equation}

     Volviendo al caso que nos incumbe, $M=\RR^2$ y podemos considerar el lagrangiano del sistema TTW
     \begin{equation}
       \mathcal{L}_k(x,y,\dot{x},\dot{y})=\tfrac{1}{2}[\dot{x}^2+\dot{y}^2+\omega^2(x^2+y^2)]+V_k(x,y).
     \end{equation}
     Tomamos ahora el potencial $V_k$ en coordenadas polares
\begin{equation*}
  V_k(r,\phi)=\frac{\alpha k^2}{2r^2\cos^2 k\phi}+\frac{\beta k^2}{2r^2\sin^2 k\phi}
\end{equation*}
y lo cambiamos a las nuevas coordenadas
\begin{equation*}
  \begin{cases}
    z=x+iy \\
    \bar{z}=x-iy,
  \end{cases}
\end{equation*}
teniendo en cuenta que $z=r e^{i\phi}$ y la fórmula de Moivre
\begin{equation*}
  z^k=r^k(\cos k\phi+i\sin k\phi),
\end{equation*}
de donde se tiene
\begin{equation*}
  \cos k\phi = \frac{z^k+\zz^k}{2|z|^k} \ \  \sin k\phi = \frac{z^k-\zz^k}{2i|z|^k}. 
\end{equation*}
Obtenemos entonces el potencial
\begin{equation}
  V_k(z,\zz)=\frac{2\alpha k^2 z^{k-1}\zz^{k-1}}{(z^k+\zz^k)^2}-\frac{2\beta k^2 z^{k-1}\zz^{k-1}}{(z^k-\zz^k)^2}.
\end{equation}
De modo que el lagrangiano del sistema TTW se expresa en las coordenadas complejas en la forma
\begin{equation}
  \mathcal{L}_k(z,\zz,\dot{z},\dot{\zz})=\tfrac{1}{2}(\dot{z}\dot{\zz}+\omega^2 z \zz) + V_k(z,\zz).
\end{equation}
Ahora, teniendo en cuenta que, en este caso,
\begin{equation*}
  \begin{cases}
    p_x=\dot{x}\\
    p_y=\dot{y},
  \end{cases}
\end{equation*}
tenemos los nuevos momentos canónicos conjugados
\begin{equation*}
  \begin{cases}
    p_z=\frac{\partial \mathcal{L}_k }{\partial \dot{z}}=\tfrac{1}{2} \dot{\zz}=\tfrac{1}{2}(\dot{x}-i\dot{y})=\tfrac{1}{2}(p_x-ip_y),\\
    p_{\zz}=\frac{\partial \mathcal{L}_k }{\partial \dot{\zz}}=\tfrac{1}{2} \dot{z}=\tfrac{1}{2}(\dot{x}+i\dot{y})=\tfrac{1}{2}(p_x+ip_y).
  \end{cases}
\end{equation*}
Vemos entonces que el hamiltoniano de TTW se escribe en las coordenadas complejas en la forma
\begin{equation*}
  H_k(z,\zz,p_z,p_{\zz})=\tfrac{1}{2}(p_zp_{\zz}+\omega^2 z \zz) + V_k(z,\zz).
\end{equation*}
Nótese también que podemos seguir pensando en las coordenadas $(z,\zz,p_z,p_{\zz})$ como coordenadas de Darboux, en efecto, se comprueba fácilmente que
\begin{equation*}
  \dd p_z \wedge \dd z + \dd p_{\zz} \wedge \dd \zz=\dd p_x \wedge \dd x + \dd p_y \wedge \dd y.
\end{equation*}
Finalmente, nótese que si $k=1$ recuperamos el oscilador armónico con términos centrífugos
\begin{equation*}
  V_1=\frac{2\alpha}{(z+\zz)^2}-\frac{2\beta}{(z-\zz)^2},
\end{equation*}
y, cuando $k$ sea entero el potencial será una función racional
\begin{align*}
  V_k=2k^2(x^2+y^2)^{k-1}\left[ \frac{\alpha}{\left( (x+iy)^k+(x-iy)^k \right)^2}-\frac{\beta}{\left( (x+iy)^k-(x-iy)^k \right)^2} \right].
\end{align*}

\section{El hamiltoniano de TTW en unas nuevas coordenadas}
Haciendo ahora otro cambio a coordenadas reales
\begin{equation*}
  \begin{cases}
    u=\tfrac{1}{2}(z^k+\zz^k), \\
    v=\tfrac{1}{2i}(z^k-\zz^k)
  \end{cases}
\end{equation*}
con momentos dados por
\begin{equation*}
  \begin{cases}
    p_z=\tfrac{k}{2}z^{k-1}(p_u-ip_v), \\
    p_{\zz}=\tfrac{k}{2}\zz^{k-1}(p_u+ip_v),
  \end{cases}
\end{equation*}
el hamiltoniano de TTW queda en la forma
\begin{equation}
  H_k(u,v,p_u,p_v)=\frac{k^2}{4}(u^2+v^2)^{\frac{k-1}{k}}\left[ \frac{1}{2}(p_u^2+p_v^2) \frac{2\omega^2}{k^2}(u^2+v^2)^{\frac{2}{k}-1}+\frac{2\alpha}{u^2}+ \frac{2\beta}{v^2}\right]. 
\end{equation}
Si se elimina el factor global $\frac{k^2}{4}(u^2+v^2)^{\frac{k-1}{k}}$, lo que obtenemos es un oscilador no lineal con términos centrífugos. En las siguientes secciones haremos un estudio y una interpretación del hamiltoniano de TTW en este nuevo aspecto.

\end{document}



